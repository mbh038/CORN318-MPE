% Options for packages loaded elsewhere
\PassOptionsToPackage{unicode}{hyperref}
\PassOptionsToPackage{hyphens}{url}
\PassOptionsToPackage{dvipsnames,svgnames,x11names}{xcolor}
%
\documentclass[
  letterpaper,
  DIV=11,
  numbers=noendperiod]{scrreprt}

\usepackage{amsmath,amssymb}
\usepackage{iftex}
\ifPDFTeX
  \usepackage[T1]{fontenc}
  \usepackage[utf8]{inputenc}
  \usepackage{textcomp} % provide euro and other symbols
\else % if luatex or xetex
  \usepackage{unicode-math}
  \defaultfontfeatures{Scale=MatchLowercase}
  \defaultfontfeatures[\rmfamily]{Ligatures=TeX,Scale=1}
\fi
\usepackage{lmodern}
\ifPDFTeX\else  
    % xetex/luatex font selection
\fi
% Use upquote if available, for straight quotes in verbatim environments
\IfFileExists{upquote.sty}{\usepackage{upquote}}{}
\IfFileExists{microtype.sty}{% use microtype if available
  \usepackage[]{microtype}
  \UseMicrotypeSet[protrusion]{basicmath} % disable protrusion for tt fonts
}{}
\makeatletter
\@ifundefined{KOMAClassName}{% if non-KOMA class
  \IfFileExists{parskip.sty}{%
    \usepackage{parskip}
  }{% else
    \setlength{\parindent}{0pt}
    \setlength{\parskip}{6pt plus 2pt minus 1pt}}
}{% if KOMA class
  \KOMAoptions{parskip=half}}
\makeatother
\usepackage{xcolor}
\setlength{\emergencystretch}{3em} % prevent overfull lines
\setcounter{secnumdepth}{5}
% Make \paragraph and \subparagraph free-standing
\makeatletter
\ifx\paragraph\undefined\else
  \let\oldparagraph\paragraph
  \renewcommand{\paragraph}{
    \@ifstar
      \xxxParagraphStar
      \xxxParagraphNoStar
  }
  \newcommand{\xxxParagraphStar}[1]{\oldparagraph*{#1}\mbox{}}
  \newcommand{\xxxParagraphNoStar}[1]{\oldparagraph{#1}\mbox{}}
\fi
\ifx\subparagraph\undefined\else
  \let\oldsubparagraph\subparagraph
  \renewcommand{\subparagraph}{
    \@ifstar
      \xxxSubParagraphStar
      \xxxSubParagraphNoStar
  }
  \newcommand{\xxxSubParagraphStar}[1]{\oldsubparagraph*{#1}\mbox{}}
  \newcommand{\xxxSubParagraphNoStar}[1]{\oldsubparagraph{#1}\mbox{}}
\fi
\makeatother


\providecommand{\tightlist}{%
  \setlength{\itemsep}{0pt}\setlength{\parskip}{0pt}}\usepackage{longtable,booktabs,array}
\usepackage{calc} % for calculating minipage widths
% Correct order of tables after \paragraph or \subparagraph
\usepackage{etoolbox}
\makeatletter
\patchcmd\longtable{\par}{\if@noskipsec\mbox{}\fi\par}{}{}
\makeatother
% Allow footnotes in longtable head/foot
\IfFileExists{footnotehyper.sty}{\usepackage{footnotehyper}}{\usepackage{footnote}}
\makesavenoteenv{longtable}
\usepackage{graphicx}
\makeatletter
\newsavebox\pandoc@box
\newcommand*\pandocbounded[1]{% scales image to fit in text height/width
  \sbox\pandoc@box{#1}%
  \Gscale@div\@tempa{\textheight}{\dimexpr\ht\pandoc@box+\dp\pandoc@box\relax}%
  \Gscale@div\@tempb{\linewidth}{\wd\pandoc@box}%
  \ifdim\@tempb\p@<\@tempa\p@\let\@tempa\@tempb\fi% select the smaller of both
  \ifdim\@tempa\p@<\p@\scalebox{\@tempa}{\usebox\pandoc@box}%
  \else\usebox{\pandoc@box}%
  \fi%
}
% Set default figure placement to htbp
\def\fps@figure{htbp}
\makeatother
% definitions for citeproc citations
\NewDocumentCommand\citeproctext{}{}
\NewDocumentCommand\citeproc{mm}{%
  \begingroup\def\citeproctext{#2}\cite{#1}\endgroup}
\makeatletter
 % allow citations to break across lines
 \let\@cite@ofmt\@firstofone
 % avoid brackets around text for \cite:
 \def\@biblabel#1{}
 \def\@cite#1#2{{#1\if@tempswa , #2\fi}}
\makeatother
\newlength{\cslhangindent}
\setlength{\cslhangindent}{1.5em}
\newlength{\csllabelwidth}
\setlength{\csllabelwidth}{3em}
\newenvironment{CSLReferences}[2] % #1 hanging-indent, #2 entry-spacing
 {\begin{list}{}{%
  \setlength{\itemindent}{0pt}
  \setlength{\leftmargin}{0pt}
  \setlength{\parsep}{0pt}
  % turn on hanging indent if param 1 is 1
  \ifodd #1
   \setlength{\leftmargin}{\cslhangindent}
   \setlength{\itemindent}{-1\cslhangindent}
  \fi
  % set entry spacing
  \setlength{\itemsep}{#2\baselineskip}}}
 {\end{list}}
\usepackage{calc}
\newcommand{\CSLBlock}[1]{\hfill\break\parbox[t]{\linewidth}{\strut\ignorespaces#1\strut}}
\newcommand{\CSLLeftMargin}[1]{\parbox[t]{\csllabelwidth}{\strut#1\strut}}
\newcommand{\CSLRightInline}[1]{\parbox[t]{\linewidth - \csllabelwidth}{\strut#1\strut}}
\newcommand{\CSLIndent}[1]{\hspace{\cslhangindent}#1}

\KOMAoption{captions}{tableheading}
\makeatletter
\@ifpackageloaded{tcolorbox}{}{\usepackage[skins,breakable]{tcolorbox}}
\@ifpackageloaded{fontawesome5}{}{\usepackage{fontawesome5}}
\definecolor{quarto-callout-color}{HTML}{909090}
\definecolor{quarto-callout-note-color}{HTML}{0758E5}
\definecolor{quarto-callout-important-color}{HTML}{CC1914}
\definecolor{quarto-callout-warning-color}{HTML}{EB9113}
\definecolor{quarto-callout-tip-color}{HTML}{00A047}
\definecolor{quarto-callout-caution-color}{HTML}{FC5300}
\definecolor{quarto-callout-color-frame}{HTML}{acacac}
\definecolor{quarto-callout-note-color-frame}{HTML}{4582ec}
\definecolor{quarto-callout-important-color-frame}{HTML}{d9534f}
\definecolor{quarto-callout-warning-color-frame}{HTML}{f0ad4e}
\definecolor{quarto-callout-tip-color-frame}{HTML}{02b875}
\definecolor{quarto-callout-caution-color-frame}{HTML}{fd7e14}
\makeatother
\makeatletter
\@ifpackageloaded{bookmark}{}{\usepackage{bookmark}}
\makeatother
\makeatletter
\@ifpackageloaded{caption}{}{\usepackage{caption}}
\AtBeginDocument{%
\ifdefined\contentsname
  \renewcommand*\contentsname{Table of contents}
\else
  \newcommand\contentsname{Table of contents}
\fi
\ifdefined\listfigurename
  \renewcommand*\listfigurename{List of Figures}
\else
  \newcommand\listfigurename{List of Figures}
\fi
\ifdefined\listtablename
  \renewcommand*\listtablename{List of Tables}
\else
  \newcommand\listtablename{List of Tables}
\fi
\ifdefined\figurename
  \renewcommand*\figurename{Figure}
\else
  \newcommand\figurename{Figure}
\fi
\ifdefined\tablename
  \renewcommand*\tablename{Table}
\else
  \newcommand\tablename{Table}
\fi
}
\@ifpackageloaded{float}{}{\usepackage{float}}
\floatstyle{ruled}
\@ifundefined{c@chapter}{\newfloat{codelisting}{h}{lop}}{\newfloat{codelisting}{h}{lop}[chapter]}
\floatname{codelisting}{Listing}
\newcommand*\listoflistings{\listof{codelisting}{List of Listings}}
\makeatother
\makeatletter
\makeatother
\makeatletter
\@ifpackageloaded{caption}{}{\usepackage{caption}}
\@ifpackageloaded{subcaption}{}{\usepackage{subcaption}}
\makeatother

\usepackage{bookmark}

\IfFileExists{xurl.sty}{\usepackage{xurl}}{} % add URL line breaks if available
\urlstyle{same} % disable monospaced font for URLs
\hypersetup{
  pdftitle={CORN318 Marine Pollution and Ecotoxicology},
  pdfauthor={Michael Hunt},
  colorlinks=true,
  linkcolor={blue},
  filecolor={Maroon},
  citecolor={Blue},
  urlcolor={Blue},
  pdfcreator={LaTeX via pandoc}}


\title{CORN318 Marine Pollution and Ecotoxicology}
\author{Michael Hunt}
\date{2026-01-29}

\begin{document}
\maketitle

\renewcommand*\contentsname{Table of contents}
{
\hypersetup{linkcolor=}
\setcounter{tocdepth}{2}
\tableofcontents
}

\bookmarksetup{startatroot}

\chapter*{Preface}\label{preface}
\addcontentsline{toc}{chapter}{Preface}

\markboth{Preface}{Preface}

Welcome to CORN318 Marine Pollution and Ecotoxicology

\bookmarksetup{startatroot}

\chapter{Ocean Acidification}\label{ocean-acidification}

Ocean acidification is one of the most significant threats to coral
reefs worldwide. It is a direct consequence of increased atmospheric
carbon dioxide (CO₂) levels, which are absorbed by the oceans. This
process alters seawater chemistry, with profound impacts on marine
ecosystems, particularly coral reefs. Below is an overview of the impact
of ocean acidification on coral reefs.

Key papers for this topic include (Doney et al. 2020), (Mollica et al.
2018), (Orr et al. 2005), (Lischka et al. 2025) and (Royal Society
2005).

\begin{figure}

\centering{

\pandocbounded{\includegraphics[keepaspectratio]{Doney et al 2020 ocean CO2.png}}

}

\caption{\label{fig-doney}Decline of ocean pH}

\end{figure}%

We see from Figure~\ref{fig-doney}, taken from (Doney et al. 2020) that
ocean pH levels have dropped by about 0.1 over the last 30-40 years.

Let us explore how this happens.\\
\strut \\

\section{Seawater Chemistry}\label{seawater-chemistry}

First, we note that atmospheric CO2 levels have risen by about 30\% in
the last 70 years, from around 300 pm to (now, in 2026) around 428 ppm,
and are continuing to rise at a rate of about 2ppm per year. This rate
of increase is, if anything, becoming greater.

The best place to see this is the site of the
\href{https://gml.noaa.gov/ccgg/trends/}{Mauna Loa Observatory} in
Hawaii, where measurements of atmospheric CO2 have been made since the
1950s. From Figure~\ref{fig-mlo} we see that over this period there has
been an annual cycle superimposed on a relentlessly rising trend.

\begin{figure}

\begin{minipage}{0.50\linewidth}

\centering{

\pandocbounded{\includegraphics[keepaspectratio]{figures/co2_trend_mlo.png}}

}

\subcaption{\label{fig-mlo_monthly}Monthly variation in atmospheric CO2}

\end{minipage}%
%
\begin{minipage}{0.50\linewidth}

\centering{

\pandocbounded{\includegraphics[keepaspectratio]{figures/co2_data_mlo.png}}

}

\subcaption{\label{fig-mlo_co2_trend}Hanno}

\end{minipage}%

\caption{\label{fig-mlo}Atmospheric CO2 data from the Maua Loa
observatory in Hawaii}

\end{figure}%

The oceans absorb about 25-30\% of the CO₂ emitted into the atmosphere.
When CO₂ dissolves in seawater, it reacts with water to form carbonic
acid, which dissociates into bicarbonate ions and hydrogen ions. The
increase in atmospheric CO2 means that more CO2 is being absorbed by the
oceans which means in turn that the concentration of hydrogen ions in
the water is increasing. This means that the ocean is becoming more
acidic, with reducd pH.

At seawater pH levels (∼8), \(\ce{CO2}\) added to seawater reacts with
water to form bicarbonate \(\ce{HCO3-}\) and hydrogen ions \(\ce{H+}\):

\[\ce{CO2 + H2O -> H2CO3- + H+}\] The release of \(\ce{H+}\) acts to
increase acidity and lower seawater pH, defined as

\[\ce{pH}=-\log_{10}\ce{[H+]}\]

\begin{tcolorbox}[enhanced jigsaw, breakable, colback=white, bottomrule=.15mm, opacityback=0, coltitle=black, toprule=.15mm, titlerule=0mm, left=2mm, leftrule=.75mm, colframe=quarto-callout-note-color-frame, bottomtitle=1mm, title=\textcolor{quarto-callout-note-color}{\faInfo}\hspace{0.5em}{Note}, toptitle=1mm, colbacktitle=quarto-callout-note-color!10!white, arc=.35mm, opacitybacktitle=0.6, rightrule=.15mm]

\textbf{pH}: a measure of the acidity of water where lower pH reflects
more acidic conditions; reported on a log-scale such that a 1-unit drop
in pH is equivalent to a factor of 10 increase in acidity

\end{tcolorbox}

and lower the concentration of carbonate ions \ce{CO3^{2-}}, via

\[\ce{CO3^{2-} + H+  -> HCO3-}\] Acidification impacts will depend on
organism responses to multiple, simultaneous chemical
changes---increasing \(\ce{CO2}\), \(\ce{HCO3-}\) , and \(\ce{H+}\) and
decreasing \(\ce{CO3^{2-}}\). Many types of marine organisms that form
shells and skeletons from calcium carbonate (\(\ce{CaCO3}\)) minerals
are sensitive to acidification. The solubility of carbonate minerals,

\[\ce{CaCO3 (s) <-> CO3^{2-} + Ca2+}\]

can be expressed as a carbon saturation state

\[ \Omega = \frac{\ce{[CO3^{2-}][Ca2+]}}{K_{sp}} \]

where \(K_{sp}\) is the apparent solubility product at a given
temperature, salinity and pressure for each mineral form of
\(\ce{CaCO3}\). A value of \(\Omega\) \textless{} 1 indicates
undersaturation with respect to thermodynamic equilibrium, and under
those seawater conditions, unprotected carbonate materials will
dissolve. The multiple forms of carbonate minerals vary in
\emph{K}\textsubscript{sp} and so have different solubilities, with
calcite being less soluble than aragonite and amorphous calcium
carbonate. As \(\ce{CO2}\) increases, the \(\ce{CO3^{2-}}\)
concentration declines because of consumption with \(\ce{H+}\) (Equation
3) causing a decline in \(\Omega\).

\begin{tcolorbox}[enhanced jigsaw, breakable, colback=white, bottomrule=.15mm, opacityback=0, coltitle=black, toprule=.15mm, titlerule=0mm, left=2mm, leftrule=.75mm, colframe=quarto-callout-note-color-frame, bottomtitle=1mm, title=\textcolor{quarto-callout-note-color}{\faInfo}\hspace{0.5em}{Note}, toptitle=1mm, colbacktitle=quarto-callout-note-color!10!white, arc=.35mm, opacitybacktitle=0.6, rightrule=.15mm]

\textbf{aragonite} and \textbf{calcite} are two mineral forms of calcium
carbonate \(\ce{CaCO3}\). Both are used by marine organisms in shell and
skeleton formation by a biominerlization. Aragonite is the more soluble
of the two, with rapid dissolution kinetics

\end{tcolorbox}

Ocean surface waters exchange \(\ce{CO2}\) with the overlying atmosphere
via physical gas transfer, and the surface seawater partial pressure,
\(\ce{pCO2}\), tends to track the growth of atmospheric \(\ce{CO2}\) for
much of the global ocean, as illustrated by long-term time series
records at numerous open-ocean locations and analysis of global surface
ocean \(\ce{CO2}\) observational networks. As a result, surface pH and
\(\ce{CO3^{2-}}\) are declining, and surface ocean pH is estimated to
have dropped on average globally by approximately 0.1 units from the
preindustrial era to present, which is an ∼30\% increase in hydrogen ion
concentration.

Hence, as \(\Omega\) declines, aragonite in particular becomes more
soluble and aragonite calcareous shells of marine life forms become
vulnerable to dissolving.

Studies have already found evidence of extensive shell dissolution of
the aragonite shells of the pteropod (small snails) \emph{Limacina
helicina antarctica} in the Southern Ocean, where undersaturation (ie
\(\ce{Omega_{\text{ar}}<1})\) of aragonite is now a reality (Lischka et
al. 2025).

\section{Organismal Responses}\label{organismal-responses}

\section{Community and Ecosystem
Effects}\label{community-and-ecosystem-effects}

\bookmarksetup{startatroot}

\chapter*{References}\label{references}
\addcontentsline{toc}{chapter}{References}

\markboth{References}{References}

\phantomsection\label{refs}
\begin{CSLReferences}{1}{0}
\bibitem[\citeproctext]{ref-doney2020}
Doney, Scott C., D. Shallin Busch, Sarah R. Cooley, and Kristy J.
Kroeker. 2020. {``The Impacts of Ocean Acidification on Marine
Ecosystems and Reliant Human Communities.''} \emph{Annual Review of
Environment and Resources} 45 (1): 83--112.
\url{https://doi.org/10.1146/annurev-environ-012320-083019}.

\bibitem[\citeproctext]{ref-lischka2025}
Lischka, Silke, Jan Michels, Lennart Thomas Bach, Katharina Csenteri,
Sonja Konschak, and Stanislav N. Gorb. 2025. {``Pteropods as
Early{-}Warning Indicators of Ocean Acidification.''} \emph{Limnology
and Oceanography} 70 (6): 1651--65.
\url{https://doi.org/10.1002/lno.70079}.

\bibitem[\citeproctext]{ref-mollica2018}
Mollica, Nathaniel R, Weifu Guo, Anne L Cohen, Kuo-Fang Huang, Gavin L
Foster, Hannah K Donald, and Andrew R Solow. 2018. {``Ocean
Acidification Affects Coral Growth by Reducing Skeletal Density.''}
\emph{Proceedings of the National Academy of Sciences - PNAS} 115 (8):
1754--59. \url{https://doi.org/10.1073/pnas.1712806115}.

\bibitem[\citeproctext]{ref-orr2005}
Orr, James C., Victoria J. Fabry, Olivier Aumont, Laurent Bopp, Scott C.
Doney, Richard A. Feely, Anand Gnanadesikan, et al. 2005.
{``Anthropogenic Ocean Acidification over the Twenty-First Century and
Its Impact on Calcifying Organisms.''} \emph{Nature} 437 (7059):
681--86. \url{https://doi.org/10.1038/nature04095}.

\bibitem[\citeproctext]{ref-royalsociety2005}
Royal Society. 2005. \emph{Ocean acidification due to increasing
atmospheric carbon dioxide}. Vol. 12/05. Policy Document. London: Royal
Society.

\end{CSLReferences}




\end{document}
